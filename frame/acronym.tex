% Utility command to avoid duplicating the label
\newcommand{\acr}[2]{
    \lowercase{\def\key{#1}} % Remove the lowercase part if you want to address your glossary label to have the same capitalization as your name (first parameter)
    \newacronym{\key}{#1}{#2}
    % Dynamically create commands as shorthand for gls: \<\key> -> \gls{<\key>}
    \ifcsname\key\endcsname%
        % Do nothing to not override existing commands
    \else%
        \expandafter\xdef\csname \key\endcsname{\gls{\key}}
    \fi%
}

% Acronyms that are used quite commonly
% If they are not used, they are ignored in the glossary
\acr{API}{Application Programming Interface}
\acr{BL}{Business Logic}
\acr{CLI}{Command Line Interface}
\acr{DOM}{Document Object Model}
\acr{HTML}{HyperText Markup Language}
\acr{HTTP}{Hypertext Transfer Protocol}
\acr{IDE}{Integrated Development Environment}
\acr{IP}{Internet Protocol}
\acr{IT}{Informationstechnologie}
\acr{JEE}{Jakarte EE (ehemals Java Platform, Enterprise Edition)}
\acr{JPA}{Jakarta Persistence API}
\acr{JSON}{JavaScript Object Notation}
\acr{JVM}{Java Virtual Machine}
\acr{REST}{Representational State Transfer}
\acr{TCP}{Transmission Control Protocol}
\acr{TK}{Techniker Krankenkasse}
\acr{URI}{Uniform Resource Identifier}
\acr{URL}{Uniform Resource Locator}
\acr{YAML}{YAML Ain't Markup Language}
